\clearpage

\section*{Appendix B: Interview Transcript}
Email interview with John Britton, GitHub Staff, April 17, 2013

\subsection*{Q: How is GitHub changing the way Open Source Software (OSS) is being developed today?}
GitHub moved the open source community away from a permission culture. Prior to GitHub, contributing to an open source project was very difficult and often involved asking permission. By allowing people to fork a repository and make changes without asking for permission, GitHub has made contributing to someone else's project much easier than it has ever been. By submitting a Pull Request with suggested changes, project maintainers and contributors have a place to discuss changes to code that is already written and to refine it.

\subsection*{Q: Many new Software Developers will be starting their education/training already having experience with GitHub. How, in your opinion, will Software Engineering as a discipline (and how it is taught) change to accommodate the ideas popularized by GitHub?}

Software Engineering is moving more towards the development models seen in industry. It's much more useful for students to learn how to use the tools that they will see in industry than it is to learn how to use learning management systems and the like. I encourage teachers to assign project work in teams so that students gain experience working together and navigating the issues that arrive in a collaborative environment. While it's far from standard practice, I'm seeing more and more teachers encouraging and even requiring the use of version control, code review, and automated testing. Most of these tools have been standard in industry for quite a while, but only recently have they made their way into the classroom thanks to services like GitHub lowering the learning curve substantially.

\subsection*{Q: There are many non-code artifacts being hosted on GitHub, for example, Object Management Group Specifications, Laws and Statutes, books... etc. Do you see GitHub participation expanding in the non-code areas?}
Most definitely. There are many examples of non-code artifacts hosted on GitHub:

\begin{itemize}
\item German law\footnote{https://github.com/bundestag/gesetze}
\item Images\footnote{https://github.com/blog/817-behold-image-view-modes}
\item 3d models\footnote{https://github.com/blog/1465-stl-file-viewing}
\item Research papers\footnote{https://github.com/karthikram/smb\_git}
\end{itemize}

That's just a small sample of some interesting things on GitHub, we're seeing new uses all the time.