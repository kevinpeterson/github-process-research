% SEng 5852 - Spring 2013 White Paper Review
\documentclass{proc}
\usepackage{graphicx}
\usepackage{mathtools}
\usepackage{fixltx2e}
\usepackage{algorithmicx}
\usepackage{algpseudocode}
\MakeRobust{\Call}
\usepackage{amsmath}
\setlength\parindent{0pt}

\title{
The GitHub Open Source Development Process
\author{Kevin Peterson\\
Department of Information Technology\\
Clinical Informatics Support Systems\\
Mayo Clinic\\
200 1st Street\\
Rochester, MN  55905\\
\small \texttt{peterson.kevin@mayo.edu}
}
}

\date{}

\begin{document}
\maketitle

\begin{abstract}
Open Source Software (OSS) has produced many successful projects. The development process by which these projects are produced is generally unstructured compared to commercial software -- but a definite pattern does arrive, and is no less a pattern. GitHub, a popular OSS code hosting website, and Git, the site's SCM of choice, may have the potential to fundamentally change ths process.\\
By analyzing a subset of GitHub repositories, this report will show how GitHub has influenced some very intrinsic aspects of traditional OSS develoment, such as developer hierarchies and issue close velicity. We find that... TODO
\end{abstract}

\noindent \textbf{Keywords.} Git, GitHub, OpenSource

\section{Introduction}
Open Source Software (OSS) has fundamentally changed how we view the Software Development Process\cite{raymond1999cathedral}. OSS projects are not only viable, but successful and thriving. 

A case study of the Apache Server project\cite{mockus2000case} shows that a dedicated community can produce software that rivals or exceeds commercial offerings. This study showed that even in the unstructured context of OSS, certain structures, hierarchies, and codes of conduct emerge. Contrast the Apache Server project to a successful project on GitHub. Although the spirit and intent may be the same, the tool set is drastially different. The intent of this paper is to explore how GitHub facilitates this process, and also how it may cause it to evolve.

\subsection{Hypothesis}
\emph{Hypothesis 1: As the number of Watchers increase, the number of Repository Forks will increase.}\\
In Open Source Software, a feeling of community belonging can be intrinsicly motivating to developers\cite{lakhani2003hackers}. This feeling of belonging can be expressed passively, as a \emph{watch}, or actively, as a \emph{fork} of a repository. As \emph{watches} don't necessarily signal intent to contribute, a positive direction for an OSS project is to not only increase the number of \emph{watchers}, but also the number of \emph{forks}. As both signify a general general interest in the project, it is expected that there be some correllation between the two.\\

\emph{Hypothesis 2: As the number Repository Forks increase, the Issue Resolution Time will decrease.}\\
In OSS, a project \emph{fork} has at times carried negative connotations, and has even been referred to as a hazzard\cite{kogut2001open}. In the context of Git and GitHub, however, a \emph{fork} is a positive occurence for a project, as it signifies greater project involvement. As Git allows for easy merging of \emph{forks}, code contributions in the form of defect fixes can be incorporated quickly. Because of this, it is expected that \emph{Issue Resolution Time} will decrease as the number of \emph{forks} increases.\\

The Apache Project noted that a rapid response to problems can be obtained because OSS is not bound to release schedules in the same way commercial software is. ``Patches'' may be released at any time, by any member of the community\cite{mockus2000case}. ``Patches"" in the context of GitHub could be equated to Pull Requests.\\

This is, ultimetely, a proof of ``Linus Law"\cite{raymond1999cathedral} in the context of GitHub. In other words, the more exposure the code gets (in terms of forks), the easier bugs will be to find and fix.

\emph{Hypothesis 3: There will be more Issue Reporters than Committers by an order of magnitude}\\
Research into the Apache Server project observed that there were far more \emph{Issue Reporters} than there were code \emph{Committers}\cite{mockus2000case}. The Apache Server project is an excellent case study in this type of Developer Hierarchy. Apache is built around a small set of {\it Core Developers}, followed by {\it Defect Repairers} and {\it Defect Reporters}. Each level of this hierarchy brings with it an order of magnitude increase in number of participants. The 10-15 {\it Core Developers} contribute around 80\% of the new functionality, while the rest of the 400 code contributors focus primarily on bug fixes. The {\it Defect Reporters} were by far the largest group, with over 3000 individuals submitting bug reports.\\

Because issue reporting is of low risk to the code base, but has potentially high value, it is a perfect way for large numbers of people to contribute. It is expected that the findings of the Apache Server project research will hold true for GitHub projects as well.\\

\subsection{Research Questions}

\emph{Research Question 1: Are GitHub projects primarily focused around a small set of core Committers?}\\
A smal core of developers, however, seems to be a historial fact of OSS development\cite{mockus2000case,mockus2002two,krishnamurthy2002cave}. At the extreme of this, a study of projects on the software hosting site Sourceforge\footnote{http://sourceforge.net} found that a large percentage of OSS development is done by lone developers. Is a small core of developers intrinsic to OSS development, or have the social aspects of GitHub and the distributed nature of Git changes this?\\

\emph{Research Question 2: How is GitHub changing the OSS process?}\\
The socal aspects of GitHub are an important part of the experience\cite{dabbish2012social}. User interactions and social pressures can drive OSS development in interesting ways. Social aspects of OSS development have existed before GitHub, with mailing lists\cite{mockus2000case}, gift giving mechanisms\cite{bergquist2008power}, and other code hosting sites. What is GitHub doing to facilitate change on OSS development, and how does it see OSS development changing moving forward?\\

\emph{Research Question 3: Can GitHub be used for more that code artifacts?}\\
The Object Managment Group\textregistered\footnote{http://www.omg.org} computer industry consortium focused on technology standards. Potential standard summission teams must go through a vetting process\cite{kobryn1999uml} which allows industry representatives to provide feedback on the standard. Gathering and recording this feedback is a challenge, but GitHub's issue tracking system may be able to streamline the process.

\section{Methods}
\subsection{Collection and Storage}
Data collection was done using the GitHub REST API\footnote{http://developer.github.com/v3/}. A random selection of 1000 repositories was selected using the following algorithm:\footnote{https://github.com/kevinpeterson/github-process-research/blob/master/code/research.py}

\begin{algorithmic}
\While{$i < 1000$} 
\State $i \gets i + 1$
\State $word \gets random(word\_list)$
\State $repos \gets git\_search(word)$
\State $j \gets \Call{random}{0... \Call{size}{repos}}$
\State $repo \gets repos_j$
\State \Call{store}{repo}
\EndWhile
\end{algorithmic}

The \textsc{store} function accepts as an input a given random repository. The purpose of this function is to persist the given repository to a database for further analysis. Results were stored in a MySQL relational database management system.

\begin{figure}
\includegraphics[height=3in,width=3in]{images/er.png}
\caption{GitHub Date Entity Relationship diagram}
\label{fig:er_diagram}
\end{figure}

The database schema to store the data is decribed in \ref{fig:er_diagram}.

\subsection{Analysis}

Where the Set $S$ is the random GitHub repository sample. Each randomly collected repository $S_i$ contains a Set of attriubtes $A$. Thus, a set of attributes for a given repository can be referred to as $A_{S_i}$.

Given the total set of attributes $A$, each individual attribute set can be denoted as $A_j$. Each individual attribute processed by some aggregate function $f$, or $f(  A_j )$. For this analysis, the aggregate functions are: \textsc{mean}, \textsc{min}, \textsc{max}, \textsc{stddev}.

Results are then averaged over all repositories, yeilding:
\[ \frac{\sum\limits_{i=1}^{n} f(  A_{S{_i}j}  ) } {n} \]

The individual attrubites of $A$, or $A_j$, are described below:\\
\textit{Issues}\\
The number of issues posted to a repository's issue tracker\footnote{https://github.com/blog/411-github-issue-tracker}.

\textit{Commits}\\
The number of individual source code commits to a repository.

\textit{Closed Issues}\\
The number of issues in a repository in a state of \textsc{closed} at the time of processing.

\textit{Open Issues}\\
The number of issues in a repository in a state of \textsc{open} at the time of processing.

\textit{Issue Close Time}\\
The average time taken to close a given issue in a repository. Non-\textsc{closed} issues were not couted.

The average is calculated using the following method. Let $d(x)$ be a function that converts seconds to a Date, and let $s(x)$ be a function that converts a Date to seconds.

\[ d\left( \frac{\sum\limits_{i=1}^{n} s(i_{\textrm{closed}}) - s(i_{\textrm{opened}})  } {n} \right) \]

\textit{Issue Creators}\\

\textit{Committers}\\


\section{Results}

\subsection{Summary Statistics}
\begin{table}[!ht]
\begin{center}
\begin{tabular}{rrrrrrr}
\hline
Variable & Mean & Min & Max & Std. Dev. \\
\hline
Issues  &  30.93  &  1  &  860  &  83.18  \\
Commits  &  249.3  &  1  &  45379  &  1700.01  \\
Closed Issues  &  26.6  &  1  &  837  &  77.56  \\
Open Issues  &  11.37  &  1  &  316  &  31.63  \\
Issue Close Time  &  26.48  &  0  &  503  &  56.54  \\
Issue Creators  &  10.61  &  1  &  128  &  22.29  \\
Committers  &  4.8  &  1  &  405  &  16.39  \\

\hline
\end{tabular}
N = 100 Repositories
\caption{GitHub Summary Statistics}
\label{table:summary_stats}
\end{center}
\end{table}

Table \ref{table:summary_stats} describes summary statistics of the repository variables above. The most evident feature of the result is the large variance. In fact, for the set of tested variables $A$, each variable $A_j$ is observed to have a larger standard deviation than the variable mean.
\begin{equation}
\forall j \in A \colon \overline{j} < \sigma_{j}
\label{eq:variance}
\end{equation}
All measured variables demonstrated this characteristic.

\subsection{Hypotheses}
\emph{Hypothesis 1: As the number of Watchers increase, the number of Repository Forks will increase.}\\
A correlation coefficient of 0.938794181553 was observed betwee the number of Watchers and Forks of a repository. This suggests that a correlation between Watchers and Forks exists.
\begin{figure}
\includegraphics[height=3in,width=3in]{images/watcher_forks_scatterplot.png}
\caption{Repostitory watchers and forks}
\end{figure}
The data reinforces the hypothesis, as does other research in the area. Even though there are aguments that the number of Forks is the true measure of project success\cite{baudry2012towards}, the data suggests that the two measures are related. Also, \textit{Dabbish et al.} have noted that the number of Watches can be a key for others to gauge interest in a project, and thus determine if they should participate \cite{dabbish2013leveraging}. If the number of Watchers of a repository serves as a social cue for participation, even ``passive'' involvement with a project (Watching) could influence the ``active'' participation (Forking). This could possibly contribute to the correlation of the two variables.

\emph{Hypothesis 2: As the number Repository Forks increase, the Issue Resolution Time will decrease.}\\
No significant correlation was found between the number of repository forks and issue resolution time. It is important to note, however, that issue resolution time is not necessarily related to software product quality in the context of GitHub. We can assert that for serveral reasons. \textit{First}, GitHub issues are not necessarily product defects. Issues may be feature requests. TODOs, or general support questions. \textit{second}, GitHub projects may have primary issue tracking services elsewhere, while using the GitHub issues as secondary services or not at all. \textit{Third}, Pham \textit{et al.} explore an emerging ``testing culture'' in GitHub -- something which could have sizable impacts on software quality but wouldn't necessarily be reflected in this metric \cite{phamcreating}.
\begin{figure}
\includegraphics[height=3in,width=3in]{images/issue_close_time_forks_scatterplot.png}
\caption{Repostitory issue close time and number of forks}
\end{figure}

\emph{Hypothesis 3: There will be more Issue Reporters than Committers by an order of magnitude.}\\
\begin{figure}
\includegraphics[height=3in,width=3in]{images/issue_reporters_histogram.png}
\caption{Repostitory issue reporters}
\end{figure}
\begin{figure}
\includegraphics[height=3in,width=3in]{images/committers_histogram.png}
\caption{Repostitory committers}
\end{figure}
\begin{figure}
\includegraphics[height=3in,width=3in]{images/issue_reporters_committers_scatterplot.png}
\caption{Repostitory issue reporters and committers}
\end{figure}

\subsection{Research Questions}
\emph{Research Question 1: Are GitHub projects primarily focused around a small set of core Committers?}\\
\begin{figure}
\includegraphics[height=3in,width=3in]{images/committers_percentage_pie_chart.png}
\caption{Individual committer commit percentage per repository}
\end{figure}

\emph{Research Question 2: How is GitHub changing the OSS process?}\\
Thung \textit{et al.} show that the social coding aspects of GitHub -- specifically the developer-to-developer connectivity -- enable high collaboration rates.\cite{thung2013network} In fact, the developer connectivity was shown to be higher than SourceForge and even Facebook. In a recent interview with GitHub staff, the social aspects of GitHub are referred to as ``unique and powerful\cite{begel2013social},'' which seems to suggest that this is a strong underlying product goal for.

\emph{Research Question 3: Can GitHub be used for more that code artifacts?}\\
As part of the OMG\textregistered standardization process, the CTS2 Specification used GitHub issue tracker to track specification changes.

\section{Acknowledgments}

\bibliographystyle{plain}
\bibliography{bibliography}

\end{document}
